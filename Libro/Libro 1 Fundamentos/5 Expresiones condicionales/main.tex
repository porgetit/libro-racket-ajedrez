\documentclass[12pt, letterpaper]{article}
\usepackage[utf8]{inputenc}

\title{Condicionales: prueba 1}
\author{Kevin Esguerra Cardona}
\date{Junio 9 del 2022}

\begin{document}
    
\maketitle

- Oye\dots ¿Sabes lo que es un condicional?

- No señor, desconozco el significado de esa palabra. ¿Qué es? - El alumno supo que debía prepararse para una gran lección -

- Bueno\dots A decir verdad, es algo muy sencillo de entender. No quiero marearte con tecnicismos. Mira, un condicional es simplemente una
desición. ¿Sabes lo que es una desición verdad?

- Por supuesto.

- Entonces, eso es un condicional. Sólo que con un pequeño matíz. Verás, un condicional es una desición que toma un computador. Pero, cómo
ya bien sabrás el computador solo puede pensar en dos cosas Unos y Ceros. 

- ¿Entonces cómo puede decidir un computador algo, lo que sea? Sólo con Unos y Ceros\dots

- Bueno\dots lo que pasa es que este hace un poco de trampa. Imagina que estás en una habitación, tu cuarto por ejemplo. 

- Listo.

- Ahora, estando al lado del interruptor que controla el foco de la \newline habitación, lo presionas y se enciende la luz. Ahora puedes ver todo el
desorden. 

- Jaja ja - Rió suavemente el alumno mientras sus mejillas se coloreaban un poco.

- No te preocupes por eso. Ahora, de nuevo en nuestro experimento mental, vuelve a presionar el interruptor. Ves cómo ahora no ves nada. Esto
que estás experimentando es la escencia de un condicional. Cuando el interruptor permite el paso de energía el foco se enciende, y cuando este
no lo permite, pues\dots 

- Bien, eso lo entiendo. Pero, ¿Cómo puedo pasar esto a un computador?

- Tranquilo, vamos paso a paso. 

- Está bien\dots

- Ahora, pongámos un par de nombres a nuestros estados. Digámos que cuando el interruptor permite el paso de la corriente, llamamos a eso 
Verdadero, y cuando no se permite el paso de energía, decimos que eso es Falso. Bien, ahora supongo que podrás ver la relación de estos estados
con el sistema binario en el que piensa el computador.

- ¡Claro! Cuando el interruptor está en Verdadero, eso es un Uno, ¿Verdad?. Y cuando está en Falso, es un Cero. ¿No es así?

- Diste en el clavo. Ahora, me gustaría que estandardizaremos eso. Digamos que un Verdadero o un Uno son True. Y que un Falso o un Cero es False.
Esto nos servirá luego al momento de implementarlo en el computador.

- ¡Ok, True, Uno; Flase, Cero. Está claro.

- Excelente\dots Ahora, vamos a hacer otro ejercicio mental. Este será un poco más complicado. 

- ¡Estoy listo!

- Correcto\dots Imagina, que eres un computador. Y un sujeto te pide que evalues si un número, por ejemplo el 5, es mayor que otro, el 10.

- Pues, no lo es. Claro. ¡¿Qué es esta tontería?! ¡¿Por qué alguien preguntaría algo así?!

- No es tan simple. Recuerda que ahora mismo eres una máquina de cómputo. No puedes comunicarte con este sujeto por medio del lenguaje natural.
Debes solo usar Unos o Ceros. True o False. 

- Ah\dots es cierto. Bueno pues, le diría Cero. False. ¡Vamos que sigue siendo sencillo! 

- ¿Cinco es mayor que diez?

- No. 

- ¡No?

- ¡Digo! Este\dots False.

- ¿Doce es mayor que once?

- ¡True!

- ¿La raíz cuadrada de 81 es mayor que dos al cubo?

- True\dots Esto sigue siendo muy fácil. 

- Bueno\dots ¿Ahora que responderías a esto? ¿Doce es mayor que nueve y uno es el doble que dos?

- ¡Qué? Cómo se supone que responda a eso. ¿True, False\dots?

- Jaja, ves cómo no era tan sencillo. Antes de decirte cómo debes responder a esa pregunta, debo aclararte un par de cosas. Primero, los condicionales
son más complejos de lo que parece. Están compuestos por tres partes principales. La Condición, el lado Verdadero y el Falso. Son estas tres partes
las que hacen posible un comportamiento tan complejo en un computador cómo el tomar una desición.

- Entiendo, entonces si la condición es True se hará lo que se diga en el lado Verdadero y así, cuando este sea False, se hará lo que diga el lado
Falso. Es simple\dots

- Por supuesto, debe serlo, si no el computador no lo entendería. 

- Entiendo\dots ¿Pero eso cómo me ayuda a saber que debe responder el computador ante aquella pregunta? ¡Esto es demasiado difícil\dots!

- No temas, verás. Esa pregunta que te he propuesto es lo que se conoce cómo una condición compuesta. Y estas siguen ciertas reglas que le ayudan al
computador a decir True o False. Debo buscar algo\dots una tabla\dots ¡la tenía por aquí\dots! ¡Aquí está! Observa. - El Maestro sacó una hoja con distintas
tablas del rincón del tercer cajón derecho de su escritorio.

* Insertar tablas de verdad del AND y del OR

- Bien, ¿Cómo debería usar estas "tablas"?

- Es fácil, te lo enseñaré resolviendo la pregunta anterior. ¿Doce es mayor que nueve y uno es el doble que dos?
Cómo puedes ver ahí hay algo que nos ayuda a entener el sentido de la pregunta. Es la "y".
A estas palabras las llamamos conectores lógicos. Básicamente nos indican cual de estas tablas debemos usar. 
En este caso, debemos usar la tabla del Y o AND. El primer paso era reconocer el conector lógico dentro de la 
pregunta. Es AND. Ahora, debemos obtener los valores de verdad, True o Flase, a cada lado del conector. Algo
que hiciste maravillosamente. Son True y False, cómo bien acertaste. Pues si vemos la tabla, hay una fila donde 
el lado izquierdo es True y el lado derecho es False. Y vémos que el resultado de esta fila es False. Esto quiere
decir que la respuesta a la pregunta: ¿Doce es mayor que nueve y uno es el doble que dos? Es False. 

- Comprendo\dots Los condicionales\dots, ¡digo, las condiciones! compuestas\dots deben tener en cuenta el conector lógico
para poder dar un solo valor de verdad. 

- Correcto. 

- Maestro, cómo ya le he comentado, estoy intentando programar un juego de ajedrez para mi proyecto final.

- Comprendo, quiéres saber cómo te ayuda esto de los condicionales en la realización de tu proyecto.

- ¡Si! Por favor\dots

- Observa - El Maestro sacó de un cajón de su escritorio que estaba cerca de su regazo un tablero de ajedrez,
algo viejo con las piezas repintadas y el color blanco de las casillas un poco percudido - Este ya es algo viejo
pero nos servirá. 

- ¿Vamos a jugar? ¿Ahora?

- Algo así. Yo voy a jugar ambos lados y  tú serás mi computador, el que me dirá si puedo realizar un movimiento.

- ¡Esta bien! Adelante. 

- Bien. Computador, ¿Puedo mover peón f3?

- Peón f3... ¡True! - Dijo emocionado el alumno - Negras. 

- Veo que estás aplicando el concepto de algoritmos que te enseñé. Eso me alegra mucho. ¿Peón e6? - Dijo el Maestro
con una leve sonrisa en su rostro mientras movía al peón blanco.

- ¡False! - El alumno, presa de su entusiasmo, levanto un poco la voz sin darse cuenta. 

- Ya veo, eso sería un movimiento ilegal. Entonces, ¿Peón e5?

- ¡True! Blancas. Segunda posición.

- ¡Wow! ¿Peón g4? - El Maestro se gratifico al ver que su alumno estaba aplicando la teoría que él le había enseñado
hacerca del ajedrez. Las posiciones y la notación ahora le eran naturales. 

- ¡True! Negras. - El alumno estaba en una especie de trance, aquel que le poseía cuando algo nuevo le fascinaba. 
Sentía que soñaba despierto. 

- Dama h4 - Dijo el maestro con seguridad al saber que había terminado ambas cosas. Tanto la pertida, cómo su labor de
explicarle al alumno los condicionales.

- ¡¡Mate!! - Grito el alumno. Justo ahí se dio cuenta del estado en el que estaba y de la gran lección que había aprendido.

- Cómo verás la partida ha sido realmente corta. Es una pequeña curiosidad dentro del ajedrez. Se llama el jaque del loco.
Es conocida cómo la partida más corta posible. Espero que hayas aprendido cómo funcionan los condicionales.

- ¡Si! - Dijo el alumno emocionado. Ni siquiera el sabía las horas que le esperaban programando todos los condicionales que 
necesitaba para su proyecto. Entendía que debía considerar todos los movimientos posibles de cada pieza. Pero esto le traía
una felicidad propia solo de su edad. 

- ¡Ahora ve! Sigue programando aquel proyecto. Se que te llevará mucho tiempo terminar de programar los movimientos de las
piezas. - Dijo el maestro complacido y con un brillo en los ojos que narraba en un lenguaje indecible la alegría que 
experimentaba. - Y, cómo tarea, te recomiendo investigar sobre el NOT. Se llama operador de negación. Te será muy útil en tu
proyecto.

El alumno afirmo con la cabeza, despidiéndose con el mismo movimiento de su maestro. A paso lento, pensante, pero decidido
atravesó la puerta del aula y se dirigió por el pasillo hacia la salida sur del edificio. En ese momento no lo sabía, pero 
acababa de aprender la lección más importante que recibiría en su carrera de ingeniero. 


\end{document}